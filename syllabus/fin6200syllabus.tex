%% LyX 2.3.6.2 created this file.  For more info, see http://www.lyx.org/.
%% Do not edit unless you really know what you are doing.
\documentclass[11pt,english,letterpaper, abstracton, abstract=false, titlepage=false, headings=normal, captions=tableheading, numbers=noenddot]{scrartcl}
\usepackage{fourier}
\usepackage{berasans}
\usepackage[T1]{fontenc}
\usepackage[latin9]{inputenc}
\usepackage{geometry}
\geometry{verbose,tmargin=1in,bmargin=1.3in,lmargin=1in,rmargin=1in}
\usepackage{babel}
\usepackage{verbatim}
\usepackage{booktabs}
\usepackage{pdfpages}
\usepackage[unicode=true,pdfusetitle,
 bookmarks=true,bookmarksnumbered=false,bookmarksopen=false,
 breaklinks=true,pdfborder={0 0 0},pdfborderstyle={},backref=false,colorlinks=false]
 {hyperref}

\makeatletter

%%%%%%%%%%%%%%%%%%%%%%%%%%%%%% LyX specific LaTeX commands.
%% Because html converters don't know tabularnewline
\providecommand{\tabularnewline}{\\}
%% A simple dot to overcome graphicx limitations
\newcommand{\lyxdot}{.}


%%%%%%%%%%%%%%%%%%%%%%%%%%%%%% User specified LaTeX commands.
\usepackage{multicol}

\makeatother

\begin{document}
\title{Financial Data Analysis and Practice}
\subtitle{Babson Finance~6200}
\date{Fall 2022\thanks{Information is subject to change. This version updated \today.}}
\author{Prof.~Luke~C.D.\ Stein\\
\texttt{\href{mailto:lcdstein@babson.edu}{lcdstein@babson.edu}}}
\maketitle

\section{Course overview}

This hands-on course teaches practical skills, reviews key issues
of finance, trains students how to use different data sets for research,
provides an introduction to the use of a Bloomberg terminal, provides
an introduction to the Python computer language and environment, and
provides some introduction to different aspects of the finance profession
including ethics and compliance.

The main focus of this course is to teach students how to define and
answer financial problems using financial data and\slash or analysis.
Students will thus learn how to scope financial problems and then
find, download, and analyze financial data, as well as how to read,
interpret, and understand financial data analysis prepared by others.
Different datasets, data sources, and analysis tools such as Bloomberg,
CRSP, and WRDS will be introduced, and students are expected to be
able to use find and download data from them on their own. Programming
and statistics will also be reviewed.

In addition, students will be made aware of professional practices
and standards in different financial professions to prepare students
for rapid entry into the workplace, as well as an ethics and compliance
module. At times, the course will mimic working in an actual workplace
to provide a simulation of practical experience.

\section{Course summary}

In this hands-on project- and skill-based course, MSF students will
be exposed to key data sets across different areas, including (but
not limited to) equities, bonds, and foreign exchange across a variety
of US and international markets. Students will learn how to access,
analyze, and use different data sets, as well as to understand key
components of data infrastructure, such as the basics of company-
and security-level data identifiers, US exchanges and trading venues,
accessibility of key data sets, major data vendors, and licensing
issues. Students develop their knowledge of Bloomberg, WRDS and CRSP,
and other databases. Students will be exposed to programming in Python
and should end the course with some familiarity with Python and basic
programming skills. Students will be exposed to Bloomberg and should
end the course with some familiarity with Bloomberg and its functionality.
Specific skills in certain datasets will be covered and students will
apply these skills to specific projects. An ethics and compliance
module will be taught and covered. Interspersed throughout the course
will be topics regarding financial practice and professionalism.

\section{Course logistics}

\subsection{Class time and location}

Classes will be held on Mondays and Wednesdays in the Cutler Center
(Babson Commons~001), although some special sessions will be held
at alternate times and locations to be announced in advance. Standard
meeting times are Section~1: 8:15\textsc{am}\textendash 9:30\textsc{am},
Section~2: 11:45\textsc{am}\textendash 1:00\textsc{pm}, and Section~3:
1:15\textsc{pm}\textendash 2:30\textsc{pm.}

Except with specific permission, students may only attend their registered
section. %
\begin{comment}
Students who wish or need to attend specific course meetings remotely
should do so via Webex; the relevant links will be announced via Canvas
and Discord.
\end{comment}


\subsection{Instructor}

Luke Stein can be contacted as follows:
\begin{itemize}
\item Via the course's Discord server: This is the preferred way to contact
me for questions or comments that can be shared publicly, since other
students may benefit from my response and resulting discussion
\item Email: \texttt{\href{mailto:lcdstein@babson.edu}{lcdstein@babson.edu}}
(please include ``6200'' in subject)
\item Phone: (781)\,239-5060
\item Office hours: By Appointment in Tomasso 224 or via Webex
\end{itemize}
I generally prefer that graduate students refer to me as ``Luke''
in class and other casual settings, but appropriate professional forms
of address include ``Dr.~Stein'' and ``Prof.~Stein.''

\subsection{Prerequisites}

Registration is limited to Babson M.S.\ in Finance students; exceptions
require instructor approval.

\section{Course materials}

Outside of the classroom, course materials will distributed in three
ways
\begin{description}
\item [{Canvas}] The course \href{https://babson.instructure.com/courses/3515245}{Canvas site}
will serve mainly to distribute non-public materials (e.g., lecture
slides, homework), and to collect and grade submissions (including
peer reviews). Please ensure you are receiving Canvas announcements).
\item [{GitHub}] The course \href{https://lukestein-classes.github.io/fdap/}{GitHub page}
is where I will distribute links to useful resources, public material
(including sample \href{https://lukestein-classes.github.io/fdap/data/}{data}
and \href{https://lukestein-classes.github.io/fdap/templates/}{code}),
and maintain an \href{https://lukestein-classes.github.io/fdap/schedule}{updated course schedule}.
\item [{Discord}] The course Discord server serves as a place to chat formally
and informally about topics related to our class. Questions and comments
can engage anyone in the class, and you can tag participants by name,
section, or expertise. High quality, professional engagement through
Discord (especially answering classmates' questions) is a component
of good class participation.
\end{description}

\subsection{Technology}

In addition to technology available in the Cutler Center, and what
is required to access and engage in class (access to the web and Discord),
you will likely want to be able to access on a personal computer:
\begin{enumerate}
\item Microsoft Excel.
\item Standard, freely available data science software tools to be discussed
in class including Python (and various related packages including
pandas, NumPy, and Seaborn) and a text editor or software development
environment.
\end{enumerate}

\subsection{Books and resources}

Many resources will be useful in the course, and will discussed in
class, posted, and\slash linked online. In addition several books
will be useful; links to each are on the course \href{https://lukestein-classes.github.io/fdap/}{GitHub page}

\subsubsection{Required text}
\begin{itemize}
\item ``Introduction to Modern Statistics'' (1st~ed.), Mine �etinkaya-Rundel
and Johanna Hardin. Freely available on web, or purchase PDF or print
edition.
\end{itemize}

\subsubsection{Recommended texts}
\begin{itemize}
\item ``Think Python'' (2nd~ed.), Allen B. Downey. Freely available on
web or PDF, or purchase print edition. This is recommended for new
Python programmers, and especially those without much programming
experience generally.
\item ``A Whirlwind Tour of Python,'' Jake VanderPlas. Freely available
on web. This is recommended for more experienced programmers, or as
an efficient review\slash reference after working through more basic
Python texts.
\item ``Python Data Science Handbook,'' Jake VanderPlas. Freely available
on web, or purchase print edition. A key resource and reference for
moving from basic python into more analysis-focused tools.
\item ``Introduction to Python for Econometrics, Statistics and Data Analysis''
(5th ed.), Kevin Sheppard. Freely available PDF. Another treatments
of these topics, with paired video lessons.
\item ``Coding for Economists,'' Arthur Turrell. Free website. An application-focused
approach suitable for complete beginners who have never written any
code before, with a number of worked examples. Has useful, opinionated,
up-to-date advice on actually setting up a technology stack.
\end{itemize}

\subsubsection{Other optional texts}
\begin{itemize}
\item ``Python for Data Analysis'' (2nd~ed.), Wes McKinney. Available
only as purchased print edition. The classic book on classic tools
for using Python for Data Analysis. Goes into more detail on some
topics than PDSH, but may be harder to follow as an introduction.
\item ``Data Analytics Using Microsoft Excel With Accounting and Finance
Datasets'' (v.~2.0), Joseph M. Manzo. Available only for purchase
on web, PDF, or print. We will be using Excel throughout the course,
where I will generally assume familiarity with basic functions and
cover some more advanced ones (such as Pivot Tables) in class. Students
looking to improve their skills or for a refresher may wish to consider
this book, which has been used in FDAP in the past and (unlike many
more generic Excel books) focuses on finance applications.
\end{itemize}

\section{Grading and course requirements}

Your course grade will be calculated as follows:

\begin{center}
\begin{tabular}{lcrc}
\toprule 
Component & Type & Weight & Due Date\tabularnewline
\midrule
Class participation and professionalism & In class and on Discord & 20\% & Throughout\tabularnewline
Homework and peer review & Online submission & 10\% & Throughout\tabularnewline
Midterm project (group) & Written report & 10\% & 10/17\tabularnewline
Professional ethics module & In class and written report & 15\% & 10/24\textendash 10/31\tabularnewline
Final project (group) & In-class presentation & 15\% & 12/5\tabularnewline
Market report (individual) & Written report & 15\% & 12/7\tabularnewline
Final examination (take-home) & Online submission & 15\% & 12/7\textendash 12\tabularnewline
\midrule
\emph{Total} &  & \emph{100\%} & \tabularnewline
\bottomrule
\end{tabular}
\par\end{center}

Graded components and\slash or final course grades may be adjusted
(i.e., ``curved''), but will only be ``curved up.'' That is, any
such adjustment will guarantee that an unadjusted grade of 90\% corresponds
to an A\textendash{} or better, 80\% to B\textendash{} or better,
and 70\% to C or better. Any ``curve'' will therefore only help
you relative to a traditional numerical rubric; you will never be
``curved down.'' Information about such a ``curve'' will \emph{not}
be provided during the semester; grades will only be adjusted (at
the instructor's discretion) after the final examination.

You are responsible for retaining copies of all your submitted work
until final grades are submitted, and resubmitting or returning it
to the instructor on request.

\subsection{Class participation and professionalism}

Students should be prepared and actively participate throughout the
semester in the classroom and through the course Discord; high quality
participation demonstrates thoughtful preparation for class and a
knowledge of relevant current events, as well as engagement with in-class
material. My goal is to see overall evidence of demonstrated commitment
to learning and helping your classmates learn, which you can do in
a variety of ways; I am looking for consistency and quality rather
than quantity.

\subsection{Homework and peer review}

Approximately eight weekly homework assignments will be posted on
Canvas, where they will also be submitted. Deadlines will be indicated
on Canvas (typically Sundays at midnight). You will also need to conduct
brief peer reviews of classmates' homework submissions, which will
be assigned via Canvas. You should be prepared to discuss homework
assignments in class any time after their due date.

\emph{Homework assignments are designed principally for learning,
not assessment. You are welcome (and even encouraged!) to consult
with classmates, but you must write and submit all your own work individually.
You should cite all collaborators by name in your submission.}

\emph{You are explicitly prohibited from accessing or consulting prior-year
Finance~6200 homework assignments or solutions (including either
students' or sample solutions).}

\subsection{Midterm group project}

A midterm project will be assigned and submitted via Canvas.

\emph{The project is designed for a mix of learning and assessment.
You must complete it independently from other groups. You should not
seek assistance from anyone outside your group (whether in the class
or not) except for the instructor, including posting questions on
course-related topics or soliciting feedback on your work.}

\subsection{Professional ethics module}

You will be asked to prepare readings for discussion during an in-class
module on professional ethics. There will also be a written deliverable
addressing a practical ethical issue faced by financial professionals.
You will be graded on the quality of your preparation and contribution
to in-class discussion, and on your written deliverable.

\emph{The written deliverable is designed for a mix of learning and
assessment. You must complete it entirely independently.}

\subsection{Final group project}

A final project will be assigned via Canvas. Each group will choose
a topic relevant to the assignment, and will present their findings
in class.

\emph{The project is designed for a mix of learning and assessment.
You are welcome (and even encouraged!) to consult with classmates,
but the presentation and supporting analysis must be prepared solely
by members of your group.}

\subsection{Market report}

You will be asked to choose a financial market and write a data-based
report on it. The specific assignment will be described on Canvas,
where you will submit your report.

\emph{The project is designed for a mix of learning and assessment.
You are welcome (and even encouraged!) to consult with classmates
on your project, but you must write and submit your own work.}

\subsection{Final examination}

The class will end with an individual, take-home final examination,
which will be assigned and submitted via Canvas. You will have several
days to complete the examination.

\emph{The final examination is designed principally for assessment,
and it will be entirely ``open note''\textemdash you can use any
pre-existing resources\textemdash but you must complete it entirely
independently. You should not seek assistance from anyone (whether
in the class or not), including posting questions on course-related
topics or soliciting feedback on your work.}

\section{Course policies}

\subsection{Classroom policies and professionalism}

As a general rule, I ask that you demonstrate appropriate respect
and professionalism. More specific classroom policies will be established
and enforced only if this guiding principle proves insufficient to
ensure a productive learning environment.

Announcements will be distributed via the course's Discord server.
Please ensure that you are signed up there for your enrolled course
section, with notifications turned on (for example, for the \#announcements
channel) as appropriate. You should not assume that class is cancelled
without notice unless I (or any alternative instructor) have not arrived
by 15 minutes past the scheduled class time.

Course content, including lectures, may be copyrighted material and
students may not sell notes taken during the conduct of the course.
Course material including lecture recordings may not be distributed
except with specific permission.

\subsection{Academic integrity and ethical behavior}

In this course, you are required to abide by the College's Academic
Integrity Policies and Procedures as outlined in \href{https://www.babson.edu/code-of-ethics}{Babson's Student Code of Ethics}.
Please review the College's Student Code of Ethics in its entirety,
as it is your responsibility to take the appropriate steps to ensure
your understanding of the Code. \emph{Ignorance of the policies is
not a valid excuse for any violations.}

Academic integrity is important for two reasons. First, independent
and original scholarship ensures that students derive the most they
can from their educational experience and the pursuit of knowledge.
Second, academic misconduct violates the most fundamental values of
an intellectual community and diminishes the achievements of the entire
college community. Accordingly, Babson views academic misconduct as
one of the most serious violations of the College's expectations that
a student can commit while at Babson College. Specific behaviors that
constitute academic misconduct, as defined in the Code, are \emph{cheating,
fabrication, facilitating academic dishonesty, plagiarism, participation
in academically dishonest activities, and unauthorized collaboration}.
In the instance I am presented with evidence to suggest that you engaged
in any of these behaviors, I will refer the incident to the \href{https://www.babson.edu/student-life/community-standards/}{Office of Community Standards}
for review.

For coursework, you are required to affirm your understanding of and
commitment to the academic honesty and integrity expectations set
forth in the Code. You may be asked to write the following pledge:
\begin{quote}
\textquotedblleft I have abided by the Babson Code of Ethics in this
work and pledge to be better than that which would compromise my integrity.\textquotedblright{}
\end{quote}
If you have questions relative to academic integrity expectations
within the context of a particular assignment, please ask me directly.
General questions can be directed to \href{mailto:communitystandards@babson.edu}{communitystandards@babson.edu}.

\subsection{Accommodations\label{subsec:Accommodations}}

If you are a student with a documented disability on record and wish
to have a reasonable accommodation made for you in this class, please
coordinate through the \href{https://www.babson.edu/academics/advising-and-support/accessibility-services/}{Department of Accessibility Services}
promptly. Please keep in mind that accommodations cannot be provided
retroactively.

Accomodations may be appropriate in other settings, including for
illness or other personal needs; religious practice; or university-sanctioned
activities. Any such accommodation should be requested in writing
as soon as possible (ideally at least one week in advance).

\section{Course schedule}

This schedule should be considered \emph{preliminary}, and will change
during the semester. Topics and material may change based on class
pace and interest.

\pagebreak{}

\includepdf[pages=-]{fin6200schedule}
\end{document}
